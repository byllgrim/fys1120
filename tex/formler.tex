\textbf{ADVARSEL!}

Det kan finnes feil.

Det er gjort forenklinger.
f.eks. $W = U$ når realiteten er $W = \Delta K + \Delta U$.
Forenklinger holder for oppgavene vi gjør, men gjelder ikke generelt.

Variabelnavn brukes som synomymer.
f.eks. $s = d = r$ er alle lik strekning.

\begin{description}
  \item [Sentripetalakselerasjon] $$a = \frac{v^2}{r}$$
  \item [Coulombs lov] $$\mathbf{F} = k\ \frac{qQ}{r^2}\ \mathbf{\hat{r}}$$
  \item [Coulombs konstant] $$k = \frac{1}{4\pi\epsilon_0} = 8.99\cdot 10^9$$
  \item [Enhetsvektor] $$\mathbf{\hat{r}} = \frac{\mathbf{r}}{r}$$
  \item [E-felt] $$\mathbf{E} = \frac{\mathbf{F}}{q}$$
  \item [Arbeid] $$W = Fs$$
  \item [Arbeid, bevaring av energi] $$W = \Delta K + \Delta U$$
  \item [Elektrisk potensiale] $$V = \frac{U}{q} = \frac{W}{q}
                               = \frac{Fd}{q} = Ed$$
  \item [Ladningstetthet] $$\lambda = \frac{Q}{L} = \text{linjetetthet}$$
                          $$\sigma = \frac{Q}{A} = \text{arealtetthet}$$
                          $$\rho = \frac{Q}{V} = \text{volumtetthet}$$
  \item [Gauss' lov] $$\Phi = \int\mathbf{E}\cdot\mathbf{dA}
                     = \frac{q}{\epsilon_0}$$
  \item [Homogent e-felt?] $$\int\mathbf{E}\cdot\mathbf{dA}
                           = EA = \frac{q}{\epsilon_0}$$
                           $$\implies E = \frac{q}{\epsilon_0 A}
                           = \frac{\sigma A}{\epsilon_0 A}
                           = \frac{\sigma}{\epsilon_0}$$
  \item [Elektrisk dipolmoment] $$\mathbf{p} = q \mathbf{d}$$
  \item [Dipol pot. energi] $$$$
  \item [Energi i kondensator]
         $$U = \frac{1}{2}CV^2
         = \frac{1}{2}C\left(\frac{Q}{C}\right)^2
         = \frac{1}{2}\frac{Q^2}{C}$$
  \item [] $$$$
\end{description}
