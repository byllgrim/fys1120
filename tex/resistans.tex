Resistans, motstand som vi kjenner det.

La oss ta utgangspunkt i resistivitet.
$$\rho = \frac{E}{J}
       = \frac{V}{J L}
       = \frac{A V}{I L}$$
$$\implies
  V = \frac{\rho L}{A} I$$
$$\implies
  R = \frac{\rho L}{A}$$

Her brukes det at
$$V = E L \implies E = \frac{V}{L}$$
$$J = \frac{I}{A}$$
$$V = R I$$

Dette gir mening fordi resistivitet er $\Omega$m,
og $\frac{L}{A}$ er $\frac{m}{m^2}$, så vi får fjernet meteren.

Hvorfor ikke bare dele på L? Fordi motstanden avhenger av tykkelsen.
