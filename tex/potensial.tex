Potensiell energi $U$ er relativt til ladningen vi betrakter.
Elektrisk potensial $V$ er en egenskap til rommet, som vi ser når vi
ser bort ifra testladningens størrelse $q$.

$$V = \frac{U}{q}$$

Når vi måler med et multimeter, måler vi potensialforskjellen mellom
to punkter a og b.
La oss kombinere $v = \frac{U}{q}$ med ligningen for potensiell energi.

$$\int_a^b \mathbf{E}\cdot\mathbf{dr}
  = V_a - V_B$$
