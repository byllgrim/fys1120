Når en dipol plasseres i et e-felt, vil like ladninger frastøte hverandre,
og dipolen vil rotere til den peker i samme retningen som e-feltet.



\paragraph{Potensiell energi} \hfill \\
Den potensielle energien $U$ til en dipol i e-felt $E$, altså arbeidet $W$
for å rotere den, er
$$U = W = Ep\cos{\theta} = -\mathbf{p}\cdot\mathbf{E}$$

$$U = W
  = Fs
  = F\left(\frac{d}{2} - \frac{d}{2}\cos{\theta}\right)
  = Eq\left(\frac{d}{2} - \frac{d}{2}\cos{\theta}\right)$$
TODO?



\paragraph{Dreiemoment (torque)} \hfill \\
Dreiemomentet $\tau$ er gitt ved
$$\tau = Ep\sin{\theta} = \mathbf{E}\times\mathbf{p}$$
