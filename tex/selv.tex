På lik linje med gjensidig induktans,
kan en krets indusere en spenning i seg selv.
Etter Lenz' lov, motvirker den det påtrykte potensialet.

I en krets med $N$ viklinger, er selvinduktans $L$ gitt ved
$$L = \frac{N\Phi_B}{i}$$

Man kan reorganisere og tidsderivere for å finne
$$L\frac{di}{dt} = N\frac{d\Phi_B}{dt}$$
Det gjenkjenner vi som Faraday?
$$\emf = -N\frac{d\Phi_B}{dt}
  = -L\frac{di}{dt}$$
