Tenk på to strømsløyfer, 1 og 2, plassert langs samme symmetriakse.

Hvis feltet i sløyfe 1 forandres, vil fluksen forandres i sløyfe 2
og en spenning induseres.

Faradays lov for den sløyfen
$$\emf_2 = -N_2 \frac{d\Phi_{B2}}{dt}$$

Den gjensidige induktansen $M_{21}$ (på 2 fra 1?) gir forholdet
$$N_2\Phi_{B2} = M_{21}i_1$$

Begge sidene kan deriveres mhp $dt$
$$N_2\frac{d\Phi_{B2}}{dt} = M_{21}\frac{di_1}{dt}$$

Setter vi det inn i faradays lov
$$\emf_2 = - M_{21}\frac{di_1}{dt}$$



\paragraph{Gjengjeldende scenario} \hfill \\
Hvis B-feltet endres i den andre sløyfen istedenfor, får vi en
gjensidig induktans $M_{12}$.
Den viser seg å være lik $M_{21}$.
$$M = M_{12} = M_{21}$$

Så gjensidig indusert emf er
$$\emf_2 = -M\frac{di_1}{dt}, \qquad
  \emf_1 = -M\frac{di_2}{dt}$$
Hvor gjensidig induktans er
$$M = \frac{N_2\Phi_{B2}}{i_1}
    = \frac{N_1\Phi_{B1}}{i_2}$$



\paragraph{Ulemper og fordeler} \hfill \\
Gjensidig induktans kan skape forstyrrelser i kretser.

I transformatorer brukes gjensidig induktans til å heve eller senke spenning.
