Dielektrikum i e-felt blir polarisert.

Eksempelvis, for dielektrikum mellom kondensatorplater:
Ladningen $Q_i$ på hver side av dielektrikumet kan finnes ved å
sammenligne kapasitansen og e-feltet før og etter innsettingen av materialet.

$$Q_0 = C_0 V_0 = C_0 E_0 d$$
$$Q_1 = C_1 V_1 = C_1 E_1 d$$

$Q_i$ er da forksjellen mellom disse ladningene.
$$Q_i = Q_0 - Q_1$$

\textbf{NB!} Ladningen på kondensatoren ble ikke nødvendigvis forandret.
Vi ser på nettoladningen.


\paragraph{Alternativt (fra et løsningsforslag)} \hfill \\
$$Q_i = Q\left( 1 - \frac{1}{\epsilon_r} \right)$$

Ladningen i dielektrikumet er forskjellen mellom den originale
ladningen $Q$,
og den reduserte (netto)ladningen $\frac{Q}{\epsilon_r}$ mellom platene.
