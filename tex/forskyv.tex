Amperes lov $\oint\vec{B}\cdot d\vec{l}=\mu_0I_{\text{encl}}$ er ufullstendig.

For en kondensator, har man en strøm $i_c$ inn til kondensatorplaten.
En integrasjonsvei rundt denne strømmen gir $\mu_0i_C$,
som forventet av Amperes lov.
Men gjør man det samme mellom platene, så er $I_{\text{encl}}=0$.
Dette er motsigende (så lenge magnetfeltet er likt).

Fra ligning for kapasitet C finner vi ladning q, og derifra strøm $i_C$,
som må være lik \emph{"strømmen"} mellom platene $i_D$.
$$q = Cv
  = \frac{\emf A}{d}(Ed)
  = \emf AE
  = \emf \Phi_E$$
$$i_C = \frac{dq}{dt}
  = \emf \frac{d\Phi_E}{dt}
  = i_D$$

Utvidelsen av Amperes lov er da
$$\oint\vec{B}\cdot d\vec{l}=\mu_0(i_C+i_D){\text{encl}}$$

Forskyvningsstrøm er ikke en strøm av elektrisk ladning,
men et tidsvarierende E-felt.
