Forestill deg en flat leder, som en liten vegg.
Et magnetfelt står normal på denne veggen.

Ladning som strømmer gjennom, vil dyttes opp (eller ned) av magnetfeltet.
Da blir det en konsentrasjon av ladning i topp og bunn av lederen,
som medfører et elektrisk felt.

Dette E-feltet bygger seg opp til kraften fra det er lik kraften fra B-feltet.
Da vil ladning passere ukrommet gjennom.
$$F_E + F_B = 0
  \implies qE_z + qv_dB_y = 0
  \implies E_z = -v_dB_y$$



\paragraph{Ladningstetthet} \hfill \\
Strømtettheten i strømretning $J_x$ er gitt ved $J_x = nqv_d$.

Hall effekten gir da
$$nq = \frac{-J_xB_y}{E_z}$$
