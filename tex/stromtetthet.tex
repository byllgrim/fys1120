En ledning (sylindrisk kabel) har en viss evne til å lede strøm,
avhengig av hvilket materiale den er laget av.

En tykkere ledning, leder bedre.
Hvor mye strøm $I$ som kan gå, for et gitt areal $A$ (tverrsnitt),
kalles for strømtetthet $J$.
$$J = \frac{I}{A}$$

Det minner om kondensatoren evne $C$ til å holde ladning $Q$
for en gitt spenning $V$.
Evnen beskriver kun forholdet, som vil holde selvom du varierer en
av parameterne.
