Tenk på en ledning sett fra siden.
Tverrsnittet har et areal A.
Materialet har n frie ladninger q per volumenhet.
Vi ser på et lite tidsinterval dt.
Da beveges det i en liten avstand dx,
en liten mengde ladning dQ.

$$\text{Strøm}
  = I
  = \frac{dQ}{dt}
  = q \frac{dN}{dt}
  = q \frac{n\ A\ dx}{dt}
  = q\ n\ A\ v_d$$

Her brukes det at
$$\text{Strøm} = \frac{ladning}{tid}$$
$$dN = \text{antall ladninger i et lite volum}= n\ A\ dx$$
$$\frac{dx}{dt} = \frac{\text{forflytning}}{\text{tid}}
                = \text{hastighet} = v_d$$


\paragraph{Detaljer} \hfill \\
$v_d$ kalles for drifthastigheten.

n avhenger av konfigurasjonen til materialet.
Gjennomsnittsfarten til ladningene.

Ladningenes bevegelse kalles virrevandring.
Fordi de ikke beveger seg uniformt i én retning, men kræsjer frem og tilbake.
