Lenz' lov er en metode for å finne retningen på indusert spenning.
Loven sier følgende:

Retningen til effekten av magnetisk induksjon,
er slik at den motvirker årsaken til effekten.

F.eks. vil en permanent magnet som faller gjennom en spole,
indusere en strøm (og dermed magnetfelt) i spolen,
og retningen er slik at fallet bremses.
