Emnet ser ut til å ha flere tema felles med \emph{Fysikk 2} fra vgs, og
\emph{FYS1210} for oss som har hatt det.

Følgende er en oversikt over begreper vi skal lære mer om.
Mer detaljer kommer i senere seksjoner.

\paragraph{Elektrostatikk}
  Det motsatte av elektrodynamikk.
  Statisk elektrisitet handler om elektrisk ladning som står i ro
  eller beveger seg langsomt.
\paragraph{Elektrisk strøm}
  Bevegelse av elektrisk ladning.
  F.eks i form av elektroner eller ioner.
\paragraph{Elektrisk kraft}
  Kraften som tiltrekker eller frastøter ladde partikler beskrives av
  Coulumbs lov.
  Kraften mellom to ladde partikler er proporsjonal med produktet av ladningene,
  og omvendt proporsjonal med kvadradet av avstanden mellom dem.
\paragraph{Kirchhoffs lov om strømmer}
  Summen av strømmene inn i et punkt, er lik summen av strømmene ut.
\paragraph{Kirchhoffs lov om spenning}
  Summen av alle spenninger i en krets er null.
  Altså, spenningsfallet over komponentene tilsvarer spenningen fra batteriet.
\paragraph{Lineære kretser}
  Krets-parametre (motstand, induktivitet, kapasitet, osv) er konstante.
  I komponentene er forholdet mellom strøm og spenning lineært.
  Inneholder ingen ikke-lineære komponenter (forsterkere, dioder,
  transistorer, osv).
\paragraph{Elektroniske komponenter}
  Motstand, diode, transistor, spole, IC, kondensator, sensor, osv.
\paragraph{Magnetfelt}
  Magnetiske felt kan lages ved bevegelse av ladde partikler og
  forekommer i magnetiske materialer.
  Deres retning og magnitude beskrives av vektorfelt.
  To enheter brukes: tesla (magnetfelt) og ampere per meter (H-felt).
\paragraph{Amperes lov}
  Det magnetiske feltet rundt en elektrisk strøm er proporsjonal med strømmen.
  På eksamen i \emph{Fysikk 2} sitter folk med høyrehåndsregelen og
  peker med tommelen som en anvendelse av amperes lov.
\paragraph{Elektromagnetisk induksjon}
  Forandring av magnetisk fluks gjennom en krets skaper spenning.
\paragraph{Forskyvningsstrøm}
  Er ikke en strøm av ladde partikler, men et elektrisk felt som varierer i tid.
\paragraph{Vekselstrøm}
  Elektrisk strøm hvor bevegelsesretningen periodisk reverseres.
  Som produsert i en alternator (generator) ved elektromagnetisk induksjon.
\paragraph{Transiente strømmer (kompleks beskrivelse)}
  Kortvarige strømmer, som f.eks. kan komme i tillegg til en sinusformet strøm.
  Oppstår bl.a. pga. forandring i magnetisk fluks.
\paragraph{Maxwells ligninger}
  En samling av fire ligninger som beskriver sammenhengen mellom elektriske og
  magnetiske felt.
  Gauss' lov, Amperes lov, Faradays induksjonslov, magnetiske monopoler.
\paragraph{Elektromagnetiske bølger}
  Elektrisk felt som oscillerer i fase med magnetisk felt og brer seg
  som tversbølger.
  F.eks. synlig lys, radiobølger osv.
\paragraph{Stråling fra ladning i bevegelse}
  Når ladde partikler aksellereres produseres elektromagnetiske bølger.
