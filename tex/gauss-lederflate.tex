For en ladet metallkule (ekstrapoler eksempelet for andre volumer)
står e-feltet normalt ut fra hvert flatestykke dA.
Derfor trenger vi ikke å bry oss vektordelen av regnestykket
$$\iint\limits_A \mathbf{E}\cdot\mathbf{dA} = E A$$

Ved gauss' lov har vi
$$E A = \frac{q}{\epsilon_0}$$

Da vet vi e-feltet langs overflaten
$$E = \frac{q}{\epsilon_0\ A}
    = \frac{\sigma\ A}{\epsilon_0\ A}
    = \frac{\sigma}{\epsilon_0}$$
