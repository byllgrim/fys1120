Når et dielektrikum føres inn i et elektrisk felt, blir materialet polarisert.
Polariseringen medfører et elektrisk felt som motvirker det som
allerede var der.
Denne reduksjonen er knyttet til den dielektriske konstanent $k$.

\textbf{Relativ permittivitet} $\epsilon_r$ er det fresheste ordet for
dielektrisitetskonstant $k$.
Det er synonymt ($\epsilon_r = k$).

Permittivitet $\epsilon$ er produktet av den relative permittiviteten
$\epsilon_r$ og permittiviteten i vakuum $\epsilon_0$.
$\epsilon = \epsilon_r \epsilon_0$.



\paragraph{Dielektrikum mellom kondensatorplater} \hfill \\
$$C = \frac{\epsilon A}{d}
    = \frac{\epsilon_r \epsilon_0 A}{d}$$

La $C_0$ være kapasitansen mellom kondensatorplater i vakuum,
og $C$ være kapasitansen etter at et dielektrikum er innsatt.
$$C_0 = \frac{\epsilon_0 A}{d}$$
$$C = \frac{\epsilon_r \epsilon_0 A}{d}$$
Da kan man se at
$$e_r = k = \frac{C}{C_0}$$


\paragraph{Relasjon til E-felt} \hfill \\
Når dielektrikumet plasseres mellom kondensatorflatene, reduseres e-feltet.

Det nye e-feltet $E$ er det originale $E_0$ minus det fra dielektrikumet $E_d$.
$$E = E_0 - E_d
    = \frac{V_0}{d} - \frac{V_d}{d}
    = \frac{1}{d} (V_0 - V_d)
    = ...
    = \frac{\sigma}{\epsilon_0 \epsilon_r}$$

Altså er det originale e-feltet redusert med en faktor på $\epsilon_r$.

\textbf{ADVARSEL!} Jeg orker ikke regne ut for å bekrefte
om dette stemmer akkurat nå.
Regn det ut selv før du stoler på at det er sant.
