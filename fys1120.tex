\documentclass{article}
\usepackage{style}

\begin{document}
  \title{FYS1120 - Oppsummering}
  \author{Robin A. T. Pedersen}
  \maketitle
  \tableofcontents

  \section{Forord}
    \textbf{ADVARSEL!} Teksten er shi... ikke top notch.

Denne teksten er ment som et sammendrag av emnet FYS1120.

Hensikten er, for meg personlig, å få en oversikt over fagets struktur.
Hvis teksten også kan fungere som et oppslagsverk eller som generell
støttelitteratur for andre, er det vel og bra.
\textbf{Men vit at det kan finnes feil og mangler.}
Jeg er ingen autoritet i feltet og skriver dette for selv å lære faget.

Innholdet er planlagt å struktureres etter forelesningene.
Avvik vil forekomme der tema i forelesningene overlapper.

  \section{Introduksjon}
    Emnet ser ut til å ha flere tema felles med \emph{Fysikk 2} fra vgs, og
\emph{FYS1210} for oss som har hatt det.

Følgende er en oversikt over begreper vi skal lære mer om.
Mer detaljer kommer i senere seksjoner.

\paragraph{Elektrostatikk}
  Det motsatte av elektrodynamikk.
  Statisk elektrisitet handler om elektrisk ladning som står i ro
  eller beveger seg langsomt.
\paragraph{Elektrisk strøm}
  Bevegelse av elektrisk ladning.
  F.eks i form av elektroner eller ioner.
\paragraph{Elektrisk kraft}
  Kraften som tiltrekker eller frastøter ladde partikler beskrives av
  Coulumbs lov.
  Kraften mellom to ladde partikler er proporsjonal med produktet av ladningene,
  og omvendt proporsjonal med kvadradet av avstanden mellom dem.
\paragraph{Kirchhoffs lov om strømmer}
  Summen av strømmene inn i et punkt, er lik summen av strømmene ut.
\paragraph{Kirchhoffs lov om spenning}
  Summen av alle spenninger i en krets er null.
  Altså, spenningsfallet over komponentene tilsvarer spenningen fra batteriet.
\paragraph{Lineære kretser}
  Krets-parametre (motstand, induktivitet, kapasitet, osv) er konstante.
  I komponentene er forholdet mellom strøm og spenning lineært.
  Inneholder ingen ikke-lineære komponenter (forsterkere, dioder,
  transistorer, osv).
\paragraph{Elektroniske komponenter}
  Motstand, diode, transistor, spole, IC, kondensator, sensor, osv.
\paragraph{Magnetfelt}
  Magnetiske felt kan lages ved bevegelse av ladde partikler og
  forekommer i magnetiske materialer.
  Deres retning og magnitude beskrives av vektorfelt.
  To enheter brukes: tesla (magnetfelt) og ampere per meter (H-felt).
\paragraph{Amperes lov}
  Det magnetiske feltet rundt en elektrisk strøm er proporsjonal med strømmen.
  På eksamen i \emph{Fysikk 2} sitter folk med høyrehåndsregelen og
  peker med tommelen som en anvendelse av amperes lov.
\paragraph{Elektromagnetisk induksjon}
  Forandring av magnetisk fluks gjennom en krets skaper spenning.
\paragraph{Forskyvningsstrøm}
  Er ikke en strøm av ladde partikler, men et elektrisk felt som varierer i tid.
\paragraph{Vekselstrøm}
  Elektrisk strøm hvor bevegelsesretningen periodisk reverseres.
  Som produsert i en alternator (generator) ved elektromagnetisk induksjon.
\paragraph{Transiente strømmer (kompleks beskrivelse)}
  Kortvarige strømmer, som f.eks. kan komme i tillegg til en sinusformet strøm.
  Oppstår bl.a. pga. forandring i magnetisk fluks.
\paragraph{Maxwells ligninger}
  En samling av fire ligninger som beskriver sammenhengen mellom elektriske og
  magnetiske felt.
  Gauss' lov, Amperes lov, Faradays induksjonslov, magnetiske monopoler.
\paragraph{Elektromagnetiske bølger}
  Elektrisk felt som oscillerer i fase med magnetisk felt og brer seg
  som tversbølger.
  F.eks. synlig lys, radiobølger osv.
\paragraph{Stråling fra ladning i bevegelse}
  Når ladde partikler aksellereres produseres elektromagnetiske bølger.

  \section{Ladning og Coulombs lov}
    \subsection{Coulomb}
      En kilde til et elektrisk felt Q (lading) vil påføre en kraft
på en annen ladning q.

Coulombs lov sier at denne kraften er
proporsjonal med produktet av ladningene og
omvendt proporsjonal med kvadratet av avstanden.

$$F = k \frac{q Q}{r^2}$$

Kraften virker like mye på begge ladningene (etter Newtons \#3 lov).

k er Coulombs konstant $8.99e9$.

  \section{Elektriske felt}
    \subsection{generelt}
      På grunn av en eller flere ladninger virker det krefter på andre ladninger
i samme rom.

Vi bruker en \emph{testladning} $q_0$ for
og måler den elektriske kraften i et rom.

Hvis vi deler på q, få vi den delen av kraften som kun avhenger
av egenskapen til rommet.
Dette er efeltet.

$$\mathbf{E} = \frac{\mathbf{F}}{q}$$

    \subsection{E-felt og spenning}
      Vi kan relatere E-felt til spenning.

$$W = U = F d$$
$$V = \frac{U}{q}
    = \frac{F d}{q}
    = E d$$

Så E-felt kan skrives som
$$E = \frac{V}{d}$$

    \subsection{Parallelle flater?}
      Ved parallelle flater? kan vi se bort fra vektordelen
av følgende integral.

$$\int \mathbf{E} \cdot \mathbf{A}
  = \frac{q}{\epsilon_0}$$

Det gir
$$E A
  = \frac{q}{\epsilon_0}$$

E-feltet er da (husk at $q = \sigma A$):
$$E = \frac{\sigma A}{\epsilon_0 A}
    = \frac{\sigma}{\epsilon_0}$$

  \section{Dipoler}
    \subsection{Elektrisk dipol}
      En dipol er to ladde objekter, med like stor men motsatt ladning,
separert med en avstand.

Dipolmonement $p$ er et mål på hvor mye separasjon $\mathbf{d}$
av ladning $\mathbf{q}$.
$$\mathbf{p} = q\ \mathbf{d}$$
Jo mer ladning, jo større separasjon av ladning.
Jo større avstand, jo større separasjon av ladning.

Retningen på dipolmomentet er definert fra minus til positiv.

    \subsection{Dipol i e-felt}
      Når en dipol plasseres i et e-felt, vil like ladninger frastøte hverandre,
og dipolen vil rotere til den peker i samme retningen som e-feltet.



\paragraph{Potensiell energi} \hfill \\
Den potensielle energien $U$ til en dipol i e-felt $E$, altså arbeidet $W$
for å rotere den, er
$$U = W = Ep\cos{\theta} = -\mathbf{p}\cdot\mathbf{E}$$

$$U = W
  = Fs
  = F\left(\frac{d}{2} - \frac{d}{2}\cos{\theta}\right)
  = Eq\left(\frac{d}{2} - \frac{d}{2}\cos{\theta}\right)$$
TODO?



\paragraph{Dreiemoment (torque)} \hfill \\
Dreiemomentet $\tau$ er gitt ved
$$\tau = Ep\sin{\theta} = \mathbf{E}\times\mathbf{p}$$

  \section{Fluks}
    \subsection{Hva er elektrisk fluks?}
      Fluks er et mål på hvor mye av et e-felt som går gjennom en flate.
Man kan tenke på hvor mange feltlinjer fra e-feltet som går gjennom flaten.

Gjennom hvert lille stykke av arealet, regner man kun den delen som står
normalt på e-feltet.
$$d\Phi = E_\perp \cdot dA$$

Den totale fluksen er da integralet av dotproduktet av vektorene
$$\Phi = \iint\limits_A \mathbf{E} \cdot \mathbf{dA}$$

  \section{Gauss' lov}
    \subsection{Beskrivelse}
      Gauss' lov beskriver forholdet mellom e-felt og ladning.

Den totale fluksen gjennom en lukket flate,
er proporsjonal med nettoladning inni flaten.
$$\Phi = \frac{q_\text{inni}}{\epsilon_0}$$
Kombinert med formelen for fluks blir det
$$\iint\limits_A \mathbf{E}\cdot\mathbf{dA} = \frac{q}{\epsilon_0}$$

    \subsection{E-felt på lederflate}
      For en ladet metallkule (ekstrapoler eksempelet for andre volumer)
står e-feltet normalt ut fra hvert flatestykke dA.
Derfor trenger vi ikke å bry oss vektordelen av regnestykket
$$\iint\limits_A \mathbf{E}\cdot\mathbf{dA} = E A$$

Ved gauss' lov har vi
$$E A = \frac{q}{\epsilon_0}$$

Da vet vi e-feltet langs overflaten
$$E = \frac{q}{\epsilon_0\ A}
    = \frac{\sigma\ A}{\epsilon_0\ A}
    = \frac{\sigma}{\epsilon_0}$$

  \section{Elektrisk arbeid og potensiell energi}
    \subsection{Ladning i e-felt}
      $$W = F\cdot s$$

Så for en ladning som beveger seg fra a til b i et e-felt har vi:
arbeid W og potensial energi U, gitt ved
$$W_{a\to b}
  = \int_a^b q\mathbf{E}\cdot\mathbf{dr}
  = -\Delta U
  = U_a - U_b$$

Den potensielle energien er arbeidet utført fra et referansepunkt
til punktet vi måler i.

  \section{Elektrisk potensial (spennende... nei vent. spenning.)}
    \subsection{Elektrisk potensial}
      Potensiell energi $U$ er relativt til ladningen vi betrakter.
Elektrisk potensial $V$ er en egenskap til rommet, som vi ser når vi
ser bort ifra testladningens størrelse $q$.

$$V = \frac{U}{q}$$

Når vi måler med et multimeter, måler vi potensialforskjellen mellom
to punkter a og b.
La oss kombinere $v = \frac{U}{q}$ med ligningen for potensiell energi.

$$\int_a^b \mathbf{E}\cdot\mathbf{dr}
  = V_a - V_B$$

    \subsection{Ekvipotensialflater}
      Områder hvor potensialet har lik skalarverdi kalles ekvipotensialflater.
Det brukes for å visualisere elektrisk potensiale.

Linjene minner om høydelinjene på et kart.
Det elektriske feltet står normalt på disse linjene.

  \section{Kondensatorer og Kapasitans}
    \subsection{Kondensator?}
      To ledende flater separert kan, når spenning er påført, samle opp
og holde på ladning.

For en gitt spenning $V$, holder en cap (capasitor/kondensator) på
en viss mengde ladning $Q$.
Forholdet mellom hvor mye ladning den kan holde, for en gitt spenning,
sier hvor god kapasitans $C$ kondensatoren har.

$$C = \frac{Q}{V}$$

    \subsection{Parallell-plate kondensator}
      Kapasitans i en kondensator $C = Q/V$ kan uttrykkes annerledes.

Når to parallelle plater utgjør kondensatoren har vi
$$C = \frac{Q}{V}
    = \frac{\sigma A}{V}
    = \frac{\sigma A}{E d}
    = \frac{\sigma A}{\frac{\sigma}{\epsilon_0}d}
    = \frac{\epsilon_0 \sigma A}{\sigma d}
    = \frac{\epsilon_0 A}{d}$$

Det bruker at
$$Q = \sigma A$$

og at
$$E = \frac{F}{q} \implies F = q E$$
$$W = U = Fd = q E d$$
$$V = \frac{U}{q} = E d$$

og til sist (E-felt normalt på flatene)
$$\int \mathbf{E} \cdot \mathbf{dA}
  = EA
  = \frac{q}{\epsilon_0}
  = \frac{\sigma A}{\epsilon_0}
  \implies
  E = \frac{\sigma A}{\epsilon_0 A}
    = \frac{\sigma}{\epsilon_0}$$

  \section{Dielektrika}
    \subsection{dielektrikum}
      Et dielektrikum, dielektrisk materiale, er en isolator som kan polariseres.

Det inneholder mange dipoler (eller dipolmomenter) som, ved tilstedeværelsen
av et elektrisk felt, vil ordne seg slik at materialet får en
positiv og en negativ side.

    \subsection{Dielektrisitetskonstanten}
      Når et dielektrikum føres inn i et elektrisk felt, blir materialet polarisert.
Polariseringen medfører et elektrisk felt som motvirker det som
allerede var der.
Denne reduksjonen er knyttet til den dielektriske konstanent $k$.

\textbf{Relativ permittivitet} $\epsilon_r$ er det fresheste ordet for
dielektrisitetskonstant $k$.
Det er synonymt ($\epsilon_r = k$).

Permittivitet $\epsilon$ er produktet av den relative permittiviteten
$\epsilon_r$ og permittiviteten i vakuum $\epsilon_0$.
$\epsilon = \epsilon_r \epsilon_0$.



\paragraph{Dielektrikum mellom kondensatorplater} \hfill \\
$$C = \frac{\epsilon A}{d}
    = \frac{\epsilon_r \epsilon_0 A}{d}$$

La $C_0$ være kapasitansen mellom kondensatorplater i vakuum,
og $C$ være kapasitansen etter at et dielektrikum er innsatt.
$$C_0 = \frac{\epsilon_0 A}{d}$$
$$C = \frac{\epsilon_r \epsilon_0 A}{d}$$
Da kan man se at
$$e_r = k = \frac{C}{C_0}$$


\paragraph{Relasjon til E-felt} \hfill \\
Når dielektrikumet plasseres mellom kondensatorflatene, reduseres e-feltet.

Det nye e-feltet $E$ er det originale $E_0$ minus det fra dielektrikumet $E_d$.
$$E = E_0 - E_d
    = \frac{V_0}{d} - \frac{V_d}{d}
    = \frac{1}{d} (V_0 - V_d)
    = ...
    = \frac{\sigma}{\epsilon_0 \epsilon_r}$$

Altså er det originale e-feltet redusert med en faktor på $\epsilon_r$.
$$E = \frac{E_0}{e_r}$$

\textbf{ADVARSEL!} Jeg orker ikke regne ut for å bekrefte
om dette stemmer akkurat nå.
Regn det ut selv før du stoler på at det er sant.

    \subsection{Indusert ladning}
      Dielektrikum i e-felt blir polarisert.

Eksempelvis, for dielektrikum mellom kondensatorplater:
Ladningen $Q_i$ på hver side av dielektrikumet kan finnes ved å
sammenligne kapasitansen og e-feltet før og etter innsettingen av materialet.

$$Q_0 = C_0 V_0 = C_0 E_0 d$$
$$Q_1 = C_1 V_1 = C_1 E_1 d$$

$Q_i$ er da forksjellen mellom disse ladningene.
$$Q_i = Q_0 - Q_1$$

\textbf{NB!} Ladningen på kondensatoren ble ikke nødvendigvis forandret.
Vi ser på nettoladningen.


\paragraph{Alternativt (fra et løsningsforslag)} \hfill \\
$$Q_i = Q\left( 1 - \frac{1}{\epsilon_r} \right)$$

Ladningen i dielektrikumet er forskjellen mellom den originale
ladningen $Q$,
og den reduserte (netto)ladningen $\frac{Q}{\epsilon_r}$ mellom platene.

  \section{Elektrisk strøm, resistivitet og resistans}
    \subsection{Strømtetthet}
      En ledning (sylindrisk kabel) har en viss evne til å lede strøm,
avhengig av hvilket materiale den er laget av.

En tykkere ledning, leder bedre.
Hvor mye strøm $I$ som kan gå, for et gitt areal $A$ (tverrsnitt),
kalles for strømtetthet $J$.
$$J = \frac{I}{A}$$

Det minner om kondensatoren evne $C$ til å holde ladning $Q$
for en gitt spenning $V$.
Evnen beskriver kun forholdet, som vil holde selvom du varierer en
av parameterne.

    \subsection{Resistivitet}
      Resistivitet $\rho$ er et materialet sin evne til å motstå elektrisk strøm.

Strøm, i en leder, kommer av elektriske felt $E$.
Jo høyere strømtetthet $J$ i forhold til e-feltet, jo mindre resistivitet.
$$\rho = \frac{E}{J}$$
Jo lavere strøm, jo mer motstand.

Enheten for resistans er ohm-meter $\Omega$m.

    \subsection{Resistans}
      Resistans, motstand som vi kjenner det.

La oss ta utgangspunkt i resistivitet.
$$\rho = \frac{E}{J}
       = \frac{V}{J L}
       = \frac{A V}{I L}$$
$$\implies
  V = \frac{\rho L}{A} I$$
$$\implies
  R = \frac{\rho L}{A}$$

Her brukes det at
$$V = E L \implies E = \frac{V}{L}$$
$$J = \frac{I}{A}$$
$$V = R I$$

Dette gir mening fordi resistivitet er $\Omega$m,
og $\frac{L}{A}$ er $\frac{m}{m^2}$, så vi får fjernet meteren.

Hvorfor ikke bare dele på L? Fordi motstanden avhenger av tykkelsen.

    \subsection{Strøm på mikroskopisk skala}
      Tenk på en ledning sett fra siden.
Tverrsnittet har et areal A.
Materialet har n frie ladninger q per volumenhet.
Vi ser på et lite tidsinterval dt.
Da beveges det i en liten avstand dx,
en liten mengde ladning dQ.

$$\text{Strøm}
  = I
  = \frac{dQ}{dt}
  = q \frac{dN}{dt}
  = q \frac{n\ A\ dx}{dt}
  = q\ n\ A\ v_d$$

Her brukes det at
$$\text{Strøm} = \frac{ladning}{tid}$$
$$dN = \text{antall ladninger i et lite volum}= n\ A\ dx$$
$$\frac{dx}{dt} = \frac{\text{forflytning}}{\text{tid}}
                = \text{hastighet} = v_d$$


\paragraph{Detaljer} \hfill \\
$v_d$ kalles for drifthastigheten.

n avhenger av konfigurasjonen til materialet.
Gjennomsnittsfarten til ladningene.

Ladningenes bevegelse kalles virrevandring.
Fordi de ikke beveger seg uniformt i én retning, men kræsjer frem og tilbake.

  \section{Elektromotorisk spenning/kraft}
    \subsection{ems? emk? (engelsk, emf, electromotive force)}
      I likhet med en fontene som sirkulerer vann ved hjelp av en vannpumpe,
må en elektrisk krets som sirkulerer ladning få hjelp av noe som påfører
elektromotorisk spenning.

Enheten forteller hvor mye energi $J$ som øverføres per ladning $C$,
altså $\frac{J}{C}$.
Eller, med andre ord, hvor my arbeid som påføres hver enhetsladning
for å flytte den mellom to punkter.

$$V = \frac{J}{C}$$
Det er samme enhet som spenning, volt ($V = U/Q$).
Men tilsynelatende brukes symbolet $\mathcal{E}$.
$$W = q\mathcal{E} = U$$



TODO? spenning, ems og indre motstand?

  \section{Energi og effekt i kretser}
    \subsection{Elektrisk effekt}
      Hvor mye arbeid $W$ du kan gjøre er én ting,
men hvor fort $t$ kan du gjøre det?
Hvor effektiv $P$ er du?

$$P = \frac{W}{t}
    = \frac{F\ s}{t}
    = \frac{q\ E\ s}{t}
    = \frac{q\ V}{t}
    = I\ V$$

Effekt (engelsk: power) er hvor mye arbeid som utføres per tid.

Effekt forteller hvor fort energi overføres.

$$P = V\ I
    = I^2\ R
    = \frac{V^2}{R}$$

Alternative måter å skrive det på.

\end{document}
